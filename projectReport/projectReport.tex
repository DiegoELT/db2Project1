\documentclass{article}

\title{Organización de Archivos y Control de Concurrencia sobre una base de datos relacional de crímenes en Boston}
\author{Sebastián Hurtado \and Diego Linares \and Piero Marini}
\date{Septiembre 26, 2019}

\begin{document}
  
  \maketitle

  \section{Introducción}
    
    \subsection{Objetivo del Proyecto}
      La meta del presente trabajo es, en una base de datos relacional, desarrollar y probar el funcionamiento de algoritmos de almacenamiento de archivos en memoria secundaria y manejo de concurrencia. Para ello, se hizo uso de un set de data de crímenes en la ciudad de Boston, Massachusets. Sobre esta se implementaron las siguientes tres técnicas de almacenamiento:
      \begin{itemize}
        \item Random File Access
        \item Static Hashing
        \item Dynamic Hashing
      \end{itemize}
      Se realizaron pruebas, por medio de transacciones, del performance de cada una de las implementaciones para observar las mejoras tanto en tiempo de retribución, como en cantidad de accesos a disco. Se emulo así mismo, el funcionamiento de concurrencia a través de threads. 
    
    \subsection{Definiciones previas}  
  
\end{document}
